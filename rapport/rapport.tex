\documentclass[a4paper]{article}

\usepackage[utf8]{inputenc}
\usepackage[T1]{fontenc}
\usepackage{textcomp}
\usepackage{url}
\usepackage[top=2.5cm,bottom=2.5cm,right=2.5cm,left=2.5cm]{geometry}
\usepackage[french]{babel}
\usepackage[backend=biber,style=ieee]{biblatex}
\addbibresource{refs.bib}

%================infos=================
\title{Création d'un langage interprété}
\author{Franck ALONSO, CHASSAGNOL Rémi}
\date{\today}

%------------------------------------------------------------------------------%
%                                  title page                                  %
%------------------------------------------------------------------------------%

\begin{document}
\maketitle

%------------------------------------------------------------------------------%
%                                 Introduction                                 %
%------------------------------------------------------------------------------%
\section{Introduction}

\section{Mots clés}

\textbf{Français :} \textit{C++}, \textit{interpréteur}, \textit{parser},
\textit{IDE}, \textit{langage de programmation}

\textbf{Anglais :} \textit{C++}, \textit{Interpreter}, \textit{Parser},
\textit{IDE}, \textit{Programming Language}


\clearpage{}

\section{Résumé des références}

\begin{itemize}
  \item La partie consultable de \cite{flexBisonHandbook} présentent les bases
    de flex.
  \item \cite{compilerFlexBison} présente la construction d'un compilateur avec
    flex et bison. Le compilateur présenté utilise une \textbf{table des
    symboles} ainsi qu'une sorte de \textbf{byte code}. Nous avons choisi
    l'autre méthode qui consiste à utiliser un object-ABS plutot que directement
    du byte code. Article très utilisé au départ pour la mise en place du
    parseur/lexeur.
  \item \cite{compilerTICH} explication du fonctionnement d'un compilateur.
  \item \cite{compilerTILB} première version de l'article précédent.
  \item \cite{crew1997astlog} création d'un analyser syntaxique pour du C/C++:
    ASTROLOG. L'article par d'analyse syntaxique et de la construction
    d'\textbf{ABS}.
  \item \cite{cppparsing}: nous a permis d'avoir un exemple de code qui allie
    flex et bison en C++ et non en C.
  \item \cite{visser2002meta} L'objectif de l'article est de présenter
    l'utilisation des \textbf{ABS} pour de la méta programmation. Il comporte
    pas mal d'exemples sur les \textbf{ABS} donc je le trouve pertinant.
  \item \cite{gagnon1998sablecc}: \textbf{ABS} en java.
\end{itemize}

\clearpage{}

%------------------------------------------------------------------------------%
%                                 bibliography                                 %
%------------------------------------------------------------------------------%
\printbibliography[keyword={paper},title={Biliographie}]
\printbibliography[keyword={web},title={Webographie}]
\end{document}
