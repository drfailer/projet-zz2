\documentclass[a4paper]{article}%{{{

\usepackage[utf8]{inputenc}%{{{
\usepackage[T1]{fontenc}
\usepackage{textcomp}
\usepackage{url}
\usepackage{hyperref}
\usepackage[top=2.5cm,bottom=2.5cm,right=2.5cm,left=2.5cm]{geometry}
\usepackage[french]{babel}
\usepackage[backend=biber,style=ieee]{biblatex}
%\usepackage{titletoc}% http://ctan.org/pkg/titletoc
\usepackage{qtree}
\usepackage{listings}
\usepackage{xcolor}
\usepackage{setspace}
\usepackage{graphicx}
\usepackage{geometry}
\usepackage{titlesec}%}}}

\addbibresource{refs.bib}

\definecolor{codegreen}{rgb}{0,0.6,0}%{{{
\definecolor{codegray}{rgb}{0.5,0.5,0.5}
\definecolor{backcolour}{rgb}{0.95,0.95,0.92}
\definecolor{backpink}{rgb}{1,0.94,0.95}
%}}}
\lstdefinestyle{codestyle}{%{{{
    backgroundcolor=\color{backcolour},
    commentstyle=\color{codegreen},
    keywordstyle=\color{magenta},
    numberstyle=\tiny\color{codegray},
    stringstyle=\color{teal},
    basicstyle=\ttfamily\footnotesize,
    breakatwhitespace=false,
    breaklines=true,
    captionpos=b,
    keepspaces=true,
    %numbers=left,
    %numbersep=5pt,
    showspaces=false,
    showstringspaces=false,
    showtabs=false,
    tabsize=2
}
%}}}
\lstdefinestyle{grammarstyle}{%{{{
    backgroundcolor=\color{backpink},
    commentstyle=\color{codegreen},
    keywordstyle=\color{purple},
    numberstyle=\tiny\color{codegray},
    stringstyle=\color{teal},
    basicstyle=\ttfamily\footnotesize,
    breakatwhitespace=false,
    breaklines=true,
    captionpos=b,
    keepspaces=true,
    %numbers=left,
    %numbersep=5pt,
    showspaces=false,
    showstringspaces=false,
    showtabs=false,
    tabsize=2
}
%}}}
\lstnewenvironment{code}[1][]%{{{
  {\noindent\minipage{\linewidth}\medskip
   \lstset{basicstyle=\ttfamily\footnotesize,frame=single,#1,upquote=true}
    \lstset{style=codestyle}
     }
  {\endminipage}

 \lstnewenvironment{grammar}[1][]
  {\noindent\minipage{\linewidth}\medskip
   \lstset{basicstyle=\ttfamily\footnotesize,frame=single,#1,upquote=true}
   \lstset{style=grammarstyle}
   }
  {\endminipage}
%}}}
%===========style & geometry==========={{{
%\lstset{style=mystyle}

 \geometry{
 a4paper,
  left=30mm,
  right=20mm,
  top=25mm,
  bottom=25mm,
 }

 \titleformat*{\section}{\LARGE\bfseries}
 \titleformat*{\subsection}{\Large\bfseries}
%}}}
%================infos================={{{
\pagenumbering{gobble}
\begin{titlepage}

\title{Création d'un langage interprété}
\author{Franck ALONSO, CHASSAGNOL Rémi}
\date{\today}
\end{titlepage}
%}}}}}}

%------------------------------------------------------------------------------%
%                                  Title page                                  %
%------------------------------------------------------------------------------%
\begin{document}
\begin{titlepage}
    \includegraphics{img/logo_isima_inp.jpeg}
       \begin{center}
           \vspace*{1cm}

           \Huge
           \textbf{Création d'un langage Transpilé}

           \vspace{0.5cm}
           \LARGE
           Rapport d'élève ingénieur\\
           Projet de 2\up{ème} année\\
           Filière F2 : Génie Logiciel et Systèmes Informatiques

           \vspace{1.5cm}

           Présenté par : \textbf{Franck ALONSO} et \textbf{Rémi CHASSAGNOL}

           \vfill

           \vspace{0.5cm}
         \end{center}


           \large
           \noindent
           Responsable ISIMA : \hfill \textbf{Jeudi 23/03/2023}\\~\\
           \raggedleft \textbf{Projet de 60h/pers}\\~\\
           \raggedright
           Campus des Cézeaux. 1 rue de la Chébarde. TSA 60125. 63178 Aubière CEDEX\\



   \end{titlepage}

\clearpage{}


%------------------------------------------------------------------------------%
%                                Remerciements                                 %
%------------------------------------------------------------------------------%
\section*{Remerciements}

\doublespacing
\large
Nous tenons à exprimer notre profonde gratitude à :\\
- Notre encadrant et cher professeur M. Loïc YON pour al qualité de son encadrement, et pour nous avoir guidées durant toute la période du projet.\\
- Mme Murielle MOUZAT, notre professeur de communication pour son aide précieuse et indispensable pour la réussite de notre projet de 2\up{ème} année.\\

\noindent Finalement, nous exprimons nos vifs remerciements à toute personne ayant participée de près ou de loin au bon déroulement de ce projet.

\normalsize
\onehalfspacing

\clearpage{}

%------------------------------------------------------------------------------%
%                               Table of contents                              %
%------------------------------------------------------------------------------%
\pagenumbering{arabic}
\thispagestyle{empty}
\tableofcontents
\clearpage{}


%------------------------------------------------------------------------------%
%                               Table of figures                               %
%------------------------------------------------------------------------------%
%\listoffigures
\clearpage{}


%------------------------------------------------------------------------------%
%                              Résumé et Abstract                              %
%------------------------------------------------------------------------------%

\setcounter{secnumdepth}{0}
\section{Résumé}
L'objectif de ce projet est la création d'un langage \textbf{interprété} en \textbf{C++} dans le but de faire découvrir l'informatique et la programmation à des collégiens et lycéens. Ce projet rentre dans le cadre du sujet commun de la filière F2, qui concerne la conception d’outils et plus généralement en développement logiciel.\\

Le projet, réalisé sous Visual Studio Code, utilisera un \textbf{lexeur} et \textbf{parseur} pour pouvoir reconnaitre notre \textbf{}{langage de programmation} voulu.
\\~\\

\noindent
Mots-clés : \textbf{C++}, \textbf{interpréteur}, \textbf{lexeur}, \textbf{parseur},  \textbf{langage de programmation}
\\[2\baselineskip]

\section{Abstract}
The objective of this project is the creation of an \textbf{interpreted} language  in \textbf{C++} in order to introduce computer science and programming to middle and high school students. This project falls within the framework of the common subject of the F2 major, which relates to the design of tools and more generally in software development.\\

The project, coded with Visual Studio Code, uses a \textbf{lexeur} and \textbf{parser} to be able to recognize our desired \textbf{}{programming language}.
\\~\\

\noindent
Keywords : \textbf{C++}, \textbf{interpreter}, \textbf{lexer}, \textbf{parser},  \textbf{programming language}

\clearpage{}


%------------------------------------------------------------------------------%
%                                 Introduction                                 %
%------------------------------------------------------------------------------%
\section{Introduction}
\large
Dans le cardre du projet de 2\up{ème} année à l'ISIMA, nous avons choisi de réalisé un travail concernant le sujet commun de la filière F2, Génie Logiciel et Systèmes d'Informations. Le but du projet est de pouvoir montrer à des élèves de collèges et lycées ce que la filière ingénieur permet de faire.\\

Le domaine étant vaste, nous avons choisi de concevoir un langage informatique simple en C++ avec comme principal objectif de faire comprendre aux élèves la notion de fonction. Ce langage devait posséder une interface graphique, mais cette tâche s'est avérée trop ambitieuse pour un travail de 60 heures.\\

Le projet, en plus de sa valeur pédagogique, nous permet de nous familiariser avec les concepts de compilateur, interpreteur et arbre syntaxique.\\

Pour présenter ce projet, nous commencerons par la forme du langage à créer souhaité, puis nous détaillerons sa structure. Une fois familier avec les différentes étapes de son implémentation,  nous introduirons les concepts et outils informatiques qui permettent la réalisation de notre langage.\\

\normalsize
\clearpage{}


%------------------------------------------------------------------------------%
%               Résumé des réfs (à supprimer à la dernière version             %
%------------------------------------------------------------------------------%
\section{Résumé des références}

\begin{itemize}
  \item La partie consultable de \cite{flexBisonHandbook} présentent les bases
    de flex.
  \item \cite{compilerFlexBison} présente la construction d'un compilateur avec
    flex et bison. Le compilateur présenté utilise une \textbf{table des
    symboles} ainsi qu'une sorte de \textbf{byte code}. Nous avons choisi
    l'autre méthode qui consiste à utiliser un object-ABS plutot que directement
    du byte code. Article très utilisé au départ pour la mise en place du
    parseur/lexeur.
  \item \cite{flexmanual}: manuel d'utilisation de Flex.
  \item \cite{cppparsing}: nous a permis d'avoir un exemple de code qui allie
    flex et bison en C++ et non en C.


  \item \cite{compilerTICH} explication du fonctionnement d'un compilateur.
  \item \cite{compilerTILB} première version de l'article précédent.
  \item \cite{crew1997astlog} création d'un analyser syntaxique pour du C/C++:
    ASTROLOG. L'article par d'analyse syntaxique et de la construction
    d'\textbf{ABS}.

  \item \cite{visser2002meta} L'objectif de l'article est de présenter
    l'utilisation des \textbf{ABS} pour de la méta programmation. Il comporte
    pas mal d'exemples sur les \textbf{ABS} donc je le trouve pertinant.
  \item \cite{gagnon1998sablecc}: \textbf{ABS} en java.
\end{itemize}


\clearpage{}
\setcounter{secnumdepth}{1}

%------------------------------------------------------------------------------%
%                        GRAMMAIRE DE NOTRE LANGAGE                            %
%------------------------------------------------------------------------------%
\section{Conception d'un langage informatique}

\phantomsection
\subsection{Grammaire de notre langage}

Avant d'implémenter notre langage informatique, il faut avoir une idée de sa grammaire, c'est-à-dire son fonctionnement. Un de ces objectifs est d'initier les élèves de collèges et lycées à la notion de fonction informatique.\\
C'est la raison pour laquelle les opérations permises par notre langage ne seront que des fonctions.\\

\textbf{Exemple} \\
Nous disposons de 2 variables \textbf{a} et \textbf{b}.\\
La variable \textbf{a} est lue par commande de l'utilisateur tandis que la valeur 4 est assignée à \textbf{b}.\\
Nous cherchons ensuite à additionner les 2 variables avec la valeur 5. Nous avons alors la syntaxe suivante :\\
- une fonction \textit{add3} qui additionne les valeurs de ses 3 paramètres\\
- une fonction \textit{affiche} qui affiche sur l'écran le nombre passé en paramètre\\
- une fonction \textit{main}, où se trouve toutes les commandes voulues de notre programme\\

\begin{grammar}[language=C++]
fn add3(int a, int b, int c) -> int {
    return add(a, add(b, c));
}

fn affiche(int n) {
    print("nombre : ");
    print(n);
}

fn main() {
    int a;
    int b;
    read(a);
    set(b, 4);
    affiche(add3(a, b, 5));
\end{grammar}\leavevmode\newline

La particularité de cette grammaire est que les opérations arithmétiques de base sont remplacées par des fonctions :\\
\begin{center}
\begin{tabular}{ | c | }
    \hline
    \textbf{a} + \textbf{b} devient \textit{add}(\textbf{a}, \textbf{b})\\
    \hline
    \textbf{a} - \textbf{b} devient \textit{mns}(\textbf{a}, \textbf{b})\\
    \hline
    \textbf{a} $\times$ \textbf{b} devient \textit{tms}(\textbf{a}, \textbf{b})\\
    \hline
    \textbf{a} $\div$ \textbf{b} devient \textit{div}(\textbf{a}, \textbf{b})\\
    \hline
\end{tabular}
\end{center}


\clearpage{}

%------------------------------------------------------------------------------%
%                                      LEXER                                   %
%------------------------------------------------------------------------------%

\section{Le lexeur}

\phantomsection
\subsection{Définition}

 Le lexeur, ou encore appelé analyseur lexical, a pour but de transformer le texte du code source en des unités lexicales, appelées \textit{tokens}, comme expliqué par la partie consultable de  \cite{flexBisonHandbook}. \\

\textbf{Exemple} \\
    Pour l'expression simple \textbf{a = 2 * b} \\
    Les tokens apparaissant sont : \\
    \begin{center}
    \begin{tabular}{ | c | c | }
    \hline
    \textbf{Token} & \textbf{Sa nature} \\
    \hline
    a & Identificateur de variable \\
    \hline
    = & Symbole d'affectation \\
    \hline
    2 & Valeur entière \\
    \hline
    * & Opérateur de multiplication \\
    \hline
    b & Identificateur de variable \\
    \hline
\end{tabular}
\end{center}

Le lexeur a également pour rôle de supprimer les informations inutiles, généralement du caractère espace et des commentaires.\\~\\


\phantomsection
\subsection{Implémentation}

L'outil utilisé pour générer un lexeur à partir du code précédent est Flex (Fast LEXical analyser generator). Au lieu d'écrire un lexeur à partir de zéro, Flex permet d'avoir un lexeur en donnant seulement les modèles des expressions régulières ainsi que le langage de travail (c++ dans notre cas).\\
\cite{compilerFlexBison} a fourni un squelette pour la réalisation du fichier nécessaire à Flex.\\~\\
\noindent
Le fichier \textbf{main\_cpp.l} contient le code qui permet de générer le lexeur avec Flex. Les token doivent-être définis dans \textbf{main\_cpp.y} au préalable.
À noter que l'on peut utiliser la variable `yylval` pour transmettre des
éléments au parser.\\

La structure du fichier \textbf{main\_cpp.l} est la suivante :

\begin{code}
C and parser declaration

%%
Grammar rules and actions

%%
C subroutings
\end{code}\leavevmode\newline


\noindent
On peut definir des règles dans les déclarations du lexeur :

\begin{code}
%option c++ interactive noyywrap noyylineno nodefault outfile="lexer.cpp"

alpha [a-zA-Z]
digit [0-9]
int [+-]?{digit}+
float [+-]?{digit}+\.{digit}+
char '{alpha}'
identifier [a-z]({alpha}|{digit}|_)*
\end{code}\leavevmode\newline

\noindent
Concernant la ligne d'option :
\begin{itemize}
\item \textbf{c++} : indique qu'on travail avec du cpp et non du c
\item \textbf{interactive} : utile quand on utlise \textbf{std::in}. Le scanner interactif regarde plus de caractères avant de générer un token (plus lent mais permet de lutter contre les ambiguitées)
\item \textbf{noyywrap} : ne pas appeler \textbf{yywrap()} qui permet de parser plusieurs fichiers
\item \textbf{noyylineno}: désactive l'enregistrement des lignes (\textbf{yylineno})
\item \textbf{nodefault} : pas de scanner par défaut (=> on doit tout implémenter)
\item \textbf{outfile:"file.cpp"} : permet de definir le fichier de sortie
\end{itemize}\leavevmode\\[3\baselineskip]


\noindent
Dans les règles, on suit toujours le même principe, on indique les caractères à reconaitre puis on exécute du code :\\

\begin{code}
for          { AFFICHE("L_for"); return Parser::token::FOR; }
{identifier} {
  AFFICHE("L_id");
  yylval->build<std::string>(yytext);
  return Parser::token::IDENTIFIER;
}
\end{code}\leavevmode\newline

\noindent
Ici on a accès à la variable \textcolor{purple}{yylval} qui est un \textcolor{purple}{Parser::semantic\_type*} et qui possède une méthode \textcolor{purple}{build} qui nous permet de transmettre des valeurs à bison.\\
Ces valeurs sont accessible via les variables de bison: \textcolor{purple}{\$2}. La variable
\textcolor{purple}{yytext} contient le text traité par Flex. De plus, dans le code, on retourne
les \textit{tokens}. Ces \textbf{tokens} sont définis dans le fichier de bison.
\newline

La fonction appelée par défaut est \textcolor{purple}{yylex}, cependant, pour pouvoir travailler avec bison, nous devons fournir nos propres fonctions, pour ce faire on utilise la macro \textcolor{purple}{YY\_DECL}, comme expliqué dans la partie \textit{9 The Generated Scanner} du manuel pour Flex \cite{flexmanual}.

\begin{code}
#define YY_DECL int interpreter::Scanner::lex(interpreter::Parser::semantic_type *yylval)
\end{code}\leavevmode\\~\\

\clearpage{}




%------------------------------------------------------------------------------%
%                                      PARSER                                  %
%------------------------------------------------------------------------------%

\section{Le parseur}

\phantomsection
\subsection{Définition}

Également appelé analyseur syntaxique, son rôle principal est la vérification de la syntaxe du code en regroupant les tokens selon une structure suivant des rêgles syntaxiques. \\


    \textbf{Exemple} \\
    Pour l'expression simple \textbf{a = 2 * b} \\
    Les tokens apparaissant sont : \\
    \begin{center}
    \begin{tabular}{ | c | c | c | }
    \hline
    \textbf{Arbre syntaxique} & \textbf{Évaluation de 2 * b} & \textbf{Affectation de a} \\
    \hline
    \Tree[.= a  [.* 2 b ]] &
        \Tree[.= a  2*b ] &
            a = 2 * b\\
    \hline
    \end{tabular}
    \end{center}

\phantomsection
\subsection{Implémentation}

À l'instar de Flex pour le lexeur, Bison est un générateur de grammaire qui convertit une description de grammaire en un programme C++ qui analyse cette même grammaire.\\
L'article \cite{compilerFlexBison} s'est encore une fois révélé très utile pour la réalisation du fichier nécessaire à Bison\\~\\
 Le fichier \textbf{main\_cpp.y} contient le code qui permet de générer le parseur avec Bison. Toutes les règles syntaxiques qui définissent la grammaire du langage y sont comprises.
Chaque règle va contenir des blocks de code qui seront exécutés au moment où le parseur la reconnait, ce code permet de créer des objets qui formeront l'ABS (Abstract Syntaxic Tree) du programme.\\


La structure du fichier \textbf{main\_cpp.y} est identique au lexeur :

\begin{code}
C and parser declaration

%%
Grammar rules and actions

%%
C subroutings
\end{code}\leavevmode\newline

Les tokens sont définis en début de fichier avec la syntaxe suivante:

\begin{code}
%token IF ELSE FOR WHILE FN INCLUDE IN
%token <long long>  INT
%token <double>     FLOAT
%token <char>       CHAR
%token <std::string> IDENTIFIER
\end{code}\leavevmode\newline

À noter que l'on peut spécifier le type de l'élément, ce qui sera utile pour
récupérer les valeurs retournées par le lexeur.\\~\\


Bison permet de construire le parser, qui va reconnaître des éléments de syntaxe
et non pas juste des mots clés. Par exemple, on peut définir une règle pour
reconnaître une suite d'inclusion de fichiers :\\

\begin{code}[language=c++]
includes: %empty
       | INCLUDE IDENTIFIER SEMI includes
       ;
\end{code}\leavevmode\newline

Ici, on définie une règle \textbf{includes} qui décrit la syntaxe des \textit{includes}. Selon cette règle, une suite d'inclusions est soit vide, soit elle comporte une inclusion, suivit d'une suite d'inclusion (`|` signifie "ou"). Il faut noter que la syntaxe est "récursive", ce qui nous permet de définir une suite d'éléments.
Enfin, les mots en majuscule sont les tokens retournés par le lexeur.\\

On peut ajouter des blocks de codes qui seront exécutés au moment où le parseur
atteint l'élément qui précède le block. Dans l'exemple ci-dessous, le block sera appelé une fois que Bison aura parser le \textcolor{purple}{;}. À noter que l'on peut accéder aux éléments retournés par Flex ; ici, \textcolor{purple}{\$2} fait référence au second élément de la règle qui est \textcolor{purple}{IDENTIFIER}. Le type de \textcolor{purple}{IDENTIFIER} a été défini comme étant une \textcolor{purple}{std::string}. Le block de code nous permet donc de créer une nouvelle inclusion et de récupérer le nom de la bibliothèque.\\

\begin{code}[language=c++]
includes: %empty
       |
       INCLUDE IDENTIFIER SEMI
       {
         std::cout << "new include id: " << $2 << std::endl;
         pb.addInclude(std::make_shared<Include>($2));
       }
       includes
       ;
\end{code}\leavevmode\newline


Comme dit plus haut, on peut définir des types pour les tokens, ce qui permet de
récupérer des valeurs:

\begin{code}[language=c++]
value: INT {
       std::cout << "new int: " << $1 << std::endl;
       lastValue.i = $1;
       lastValueType = INT;
     } | FLOAT {
       std::cout << "new double: " << $1 << std::endl;
       lastValue.f = $1;
       lastValueType = FLT;
     } | CHAR {
       std::cout << "new char: " << $1 << std::endl;
       lastValue.c = $1;
       lastValueType = CHR;
     }
     ;
\end{code}\leavevmode\newline

Pour la génération du code, on a deux options :
\begin{itemize}
\item utiliser des instruction très simples => sorte de bytecode
\item créer un code objet où tous les éléments sont des objets.
\end{itemize}\leavevmode\\~\\

\underline{Choix de la représentation objet} :
\begin{itemize}
\item plus simple à comprendre et à visualiser
\item plus compliqué à générer: on peut générer du bytecode au fil de l'exécution du parseur, en utilisant des \textcolor{orange}{goto} pour sauter de block d'instruction en block d'instruction. Pour le code objet, les éléments à l'intérieurs des blocks doivent être créés avant le block, et le block est détecté avant les instructions, il faut donc stocker les instructions.
\end{itemize}


\clearpage{}

%------------------------------------------------------------------------------%
%                                 bibliography                                 %
%------------------------------------------------------------------------------%
\printbibliography[keyword={paper},title={Biliographie}]
\printbibliography[keyword={web},title={Webographie}]
\end{document}
